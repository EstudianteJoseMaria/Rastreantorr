

 $\ast$$\ast$$\ast$$\ast$\+Proyecto Rastreantor$\ast$$\ast$$\ast$$\ast$ 



R\+A\+S\+T\+R\+E\+A\+N\+T\+OR

{\bfseries Presentación del programa}

El programa está diseñado para poder hacer un control de productos que introduces en una base de datos y poder monitorizar y gestionar el seguimiento de éstos. Se basa en la recepción y envío de mensajes entre un servidor y el cliente a través de web sockets y mensajes Json.

$\ast$$\ast$\+Herramientas utilizadas\+: $\ast$$\ast$

Programa desarrollado con los lenguajes C++, Java\+Script y H\+T\+ML. El entorno de programación usado para compilar el código ha sido Qt5.

$\ast$$\ast$\+Funcionamiento del programa $\ast$$\ast$ \begin{DoxyVerb}- El programa funciona con una servidor alojado en un PC el cual, al estar en funcinamiento, recibe los mensajes
que los clientes a través de un navegador y web sockets le envían para poder interactuar con el programa.
- Todos los clientes pueden introducir el producto a buscar y éste se almacenará en la base de datos para su
posterior uso para procesar los datos y ir actualizándose gracias a su respectiva dirección web. 
- Todos los usuarios podrán ver los productos que están en procesamiento para ver el estado del producto, es decir,
si el producto que buscan ha subido de precio, si el usuario ha cancelado su búsqueda, etc...
\end{DoxyVerb}


$\ast$$\ast$\+Problemas que hayan surgido\+: $\ast$$\ast$


\begin{DoxyItemize}
\item Han surgido bastantes problemas durante la creación dle programa, ya que al principio, no entendía bien el funcionamiento de los mensajes Json y su estructura.
\item En el momento de crear los certificados digitales del cliente salían problemas constantes y tardé demasiado en corregirlos todos.
\end{DoxyItemize}

$\ast$$\ast$\+Opciones que hubieramos implementado\+: $\ast$$\ast$

Me hubiera gustado preparar los crawlers para que rastrearan la página web que pusiera el usuario, y poder automatizarlo. También que los clientes pudieran interactuar con los productos que estaban en busqueda.

{\bfseries Base de datos}

El servidor usa una base de datos Postgre\+S\+Ql llamada sockets.

{\bfseries Usuarios de prueba}

Se puede entrar en el programa con un usuario de prueba\+: usuario\+: prueba, contraseña\+: prueba.

{\bfseries L\+I\+C\+E\+N\+C\+I\+AS}

\begin{DoxyVerb}Creative Commons (CC - NC) 2020
Esta obra está bajo una Licencia Creative Commons No-Comercial. 
La comercialización de cualquier software  o derivado de éste puede ser castigado y se aplicarán las leyes penales
del país de origen. Qualquiera puede acceder, ver y modificar el contenido del producto, pero en ningún caso puede
distribuirlo ni comercializarlo. 

José María Puigserver Sastre
02/2020\end{DoxyVerb}
 